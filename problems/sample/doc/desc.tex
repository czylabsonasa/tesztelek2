\documentclass[12pt]{extarticle}
%\usepackage[utf8]{inputenc}
\usepackage{amsmath}
\usepackage{verbatim}
\usepackage{setspace}
\onehalfspace

\newcommand{\pn}{\par\noindent}
\newcommand{\valami}[3]{\vspace{#1cm}\pn {\bf #2} #3}
\newcommand{\feladatp}[6]{
  \vspace{0.1cm}
  \centerline {\bf #1}
  \vspace{0.2cm}
  #2 
  \vspace{0.3cm}
  \pn {\bf input}
  #3
  \vspace{0.1cm}
  \pn {\bf output}
  #4
  \vspace{0.2cm}
  \pn {\bf megjegyzés}
  #5
  \vspace{0.3cm}
  \pn {\bf input A}
  \verbatiminput{../io/#6.in}
  \vspace{0.1cm}
  \pn {\bf output A}
  \verbatiminput{../io/#6.out}
}

\newcommand{\feladatpp}[7]{
  \feladatp{#1}{#2}{#3}{#4}{#5}{#6}
  \vspace{0.3cm}
  \pn {\bf input B}
  \verbatiminput{../io/#7.in}
  \vspace{0.1cm}
  \pn {\bf output B}
  \verbatiminput{../io/#7.out}
}


\begin{document}

\feladatpp{sample}{
\pn Adott $x_1,\ldots ,x_n$ minta és $q_1,\ldots , q_m$ helyek esetén határozzuk meg 
a tapasztalati eloszlásfüggvény adott helyeken felvett értékeit ($f_1,\ldots, f_m$).
}{
%input
\pn $n\ \ m$
\pn $x_1 \ldots x_n$
\pn $q_1 \ldots q_m$
}{
%output
\pn $f_1 \ldots f_m$
}{
%megjegyzés
\pn $1<n,m<100$
\pn Az $F$ tapasztalati-eloszlásüggvény az $x$-helyen $\frac{k}{n}$-et vesz fel 
ha a mintaelemek közül pontosan $k$ darab {\it kisebb} mint $x$.
}{1}{4}

\end{document}
